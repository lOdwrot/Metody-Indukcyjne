\documentclass[12pt,a4paper]{article}
\usepackage[top=2.5cm,bottom=2.5cm,left=2.2cm,right=2.2cm]{geometry}
\usepackage{polski}
\usepackage[utf8]{inputenc}
%%\usepackage[OT4]{fontenc}
\usepackage{amsmath,amsfonts,amssymb,amsthm}
\usepackage{enumerate}
\usepackage{url}
\usepackage{multicol}
\usepackage{color}
\usepackage{graphicx} 
\usepackage{setspace}
\usepackage{float}
\usepackage{subfig}
\usepackage{listings}
\usepackage{pythonhighlight}
\usepackage{lipsum}
\usepackage{tabularx}
\usepackage{hyperref}

%\pagestyle{empty}
%WYMIARY STRONY
\topmargin -30mm
\oddsidemargin -1.7cm
\evensidemargin -1.7cm
\textwidth 180mm
\textheight 260mm
%\usepackage{psfrag}

\usepackage{amsmath}
\usepackage{amsfonts}

\usepackage{supertabular}
\usepackage{array}


\usepackage{tabularx}
\usepackage{hhline}

\newcommand{\myand}{i\ }
%\usepackage{showlabels}

\newcommand{\R}{I\!\!R} %symbol liczb rzeczywistych, dzia³a tylko w
                        %trybie matematycznym
\newtheorem{theorem}{Twierdzenie}[section] %nowe otoczenie do
                                           %sk³adania twierdzeñ

\usepackage{titlesec}
\titleformat*{\section}{\normalsize\bfseries}
\titleformat*{\subsection}{\footnotesize\bfseries}
\titleformat*{\subsubsection}{\normalsize}
\title{Wybrane metody klasteryzacji w oparciu o system R}
\date{24.04.2018}
\author{Łukasz Odwrot 218283}

%ustawianie marginesów
\usepackage{geometry}
\newgeometry{tmargin=2.5cm, bmargin=2.5cm, lmargin=2.5cm, rmargin=2.5cm}


 
 
\begin{document}
\maketitle
\thispagestyle{empty}
\newpage
\tableofcontents
\setcounter{page}{1}
\newpage

\section{Opis algorytmu knn}
Algorytm \textit{K najbliższych sąsiadów} jest jednym z prostszych algorytmów klasyfikacji. Polega on na tym, że ze zbioru uczącego wybieranych jest k najbliższych wektorów (na podstawie atrybutów i wybranej metryki). Następnie na podstawie głosowania, w którym biorą udział wyselekcjonowane wektory ustala się przynależność nowego wektora do klasy.\\
Zbadane zostaną następujące metody głosowania:
\begin{enumerate}
  \item \textit{Uniform} - waga każdego wektora jest jednakowa,
  \item \textit{Distance} - wartość głosu jest proporcjonalna do odległości, waga wynosi 1/dystans,
  \item \textit{Squared Distance} - waga wynosi $1/dystans^2$.
\end{enumerate}

Ponadto do sposobu liczenia odległości użyte zostaną następujące metryki

\begin{enumerate}
  \item \textit{Euclidean}: $\sqrt{\sum_{i=1}^k(x_i - y_i)^2} $,
  \item \textit{Manhattan}: $|\sum_{i=1}^k(x_i - y_i)| $
\end{enumerate}


\section{Badane zbiory}

Klasteryzacja badana będzie na 4 zbiorach.
\begin{figure}[H]
\centering
\includegraphics[width=1\textwidth]{dsWineCombined.png}
\caption{Rozkład cech dla zbioru Wine}
\end{figure}

\begin{figure}[H]
\centering
\includegraphics[width=1\textwidth]{dsGlassCombined.png}
\caption{Rozkład cech dla zbioru Glass}
\end{figure}

\begin{figure}[H]
\centering
\includegraphics[width=1\textwidth]{dsDiabetesCombined.png}
\caption{Rozkład cech dla zbioru Diabetes}
\end{figure}

\begin{figure}[H]
\centering
\includegraphics[width=1\textwidth]{dsKnowledgeCombined.png}
\caption{Rozkład cech dla zbioru Knowledge}
\end{figure}


\section{Wpływ doboru metryki na wyniki}

Dla każdego ze zbiorów zbadano wpływ doboru metryki na jakość klasyfikacji. Sposób głosowania dla tej próby zostanie ustawiony na \textit{squaredDistances}, dla 5 sąsiadów oraz rozmiarze kroswalidacji 5.\\
\begin{tabular}{ |p{2.5cm}||p{2.5cm}|p{2.5cm}|p{2.5cm}|p{2.5cm}| }
\hline
Metric &Accuracy & Precision & Recall & FScore \\
\hline
\multicolumn{5}{|c|}{Instanacja Wine}\\
\hline
euclidean & 0.955 & 0.959 & 0.955 & 0.955\\
manhattan & 0.972 & 0.973 & 0.972 & 0.972\\
\hline
\multicolumn{5}{|c|}{Instanacja Glass}\\ 
\hline
euclidean & 0.66 &, 0.672 & 0.668 & 0.665\\
manhattan & 0.682 & 0.684 & 0.682 & 0.677\\
\hline
\multicolumn{5}{|c|}{Instanacja Diabetes}\\  
\hline
euclidean & 0.74 & 0.651 & 0.567 & 0.606\\
manhattan & 0.742 & 0.655 & 0.552 & 0.599\\
\hline
\multicolumn{5}{|c|}{Instanacja Knowledge}\\  
\hline
euclidean & 0.813 & 0.826 & 0.813 & 0.811\\
manhattan & 0.848 & 0.854 & 0.848 & 0.848\\
\hline
\end{tabular}

\begin{figure}[H]
\centering
\includegraphics[width=1\textwidth]{MetricsWine.PNG}
\caption{Confusion Matrix dla zbioru Wine}
\end{figure}

\begin{figure}[H]
\centering
\includegraphics[width=1\textwidth]{MetricsGlass.PNG}
\caption{Confusion Matrix dla zbioru Glass}
\end{figure}

\begin{figure}[H]
\centering
\includegraphics[width=1\textwidth]{MetricsDiabetes.PNG}
\caption{Confusion Matrix dla zbioru Diabetes}
\end{figure}

\begin{figure}[H]
\centering
\includegraphics[width=1\textwidth]{MetricsKnowledge.PNG}
\caption{Confusion Matrix dla zbioru Knowledge}
\end{figure}

Dla większości badanych zbiorów metryka \textit{manhattan} daje lepsze rezultaty. Jedyny wyjątek stanowi zbiór \textit{Diabetes}, ale może to być spowodowane losowością w procesie kroswalidacji.

\section{Badanie sposobu głosowania}
Dla wszystkich badanych zbiorów przetestowane zostaną różne metody głosowania. Badanie zostanie przeprowadzone dla parametru k=5, metryce manhatan i rozmiarze kroswalidacji 5.

\begin{tabular}{ |p{3cm}||p{2cm}|p{2cm}|p{2cm}|p{2cm}| }
\hline
Voting & Accuracy & Precision & Recall & FScore \\
\hline
\hline
\multicolumn{5}{|c|}{Wine}\\
\hline
uniform & 0.966 & 0.968 & 0.966 & 0.966\\
distance & 0.972 & 0.973 & 0.972 & 0.972\\
squaredDistances & 0.972 & 0.973 & 0.972 & 0.972\\
\hline
\multicolumn{5}{|c|}{Glass}\\
\hline
uniform & 0.664 & 0.622 & 0.664 & 0.636\\
distance & 0.701 & 0.7 & 0.701 & 0.69\\
squaredDistances & 0.682 & 0.684 & 0.682 & 0.677\\
\hline
\multicolumn{5}{|c|}{Diabetes}\\
\hline
uniform & 0.75 & 0.671 & 0.556 & 0.608\\
distance & 0.747 & 0.664 & 0.56 & 0.607\\
squaredDistances & 0.742 & 0.655 & 0.552 & 0.599\\
\hline
\multicolumn{5}{|c|}{Knowledge}\\
\hline
uniform & 0.841 & 0.85 & 0.841 & 0.84\\
distance & 0.856 & 0.864 & 0.856 & 0.855\\
squaredDistances & 0.848 & 0.854 & 0.848 & 0.848\\
\hline
\end{tabular}

\begin{figure}[H]
\centering
\includegraphics[width=1\textwidth]{VotingWine.PNG}
\caption{Confusion Matrix dla zbioru Wine}
\end{figure}

\begin{figure}[H]
\centering
\includegraphics[width=1\textwidth]{VotingGlass.PNG}
\caption{Confusion Matrix dla zbioru Glass}
\end{figure}

\begin{figure}[H]
\centering
\includegraphics[width=1\textwidth]{VotingDiabetes.PNG}
\caption{Confusion Matrix dla zbioru Diabetes}
\end{figure}

\begin{figure}[H]
\centering
\includegraphics[width=1\textwidth]{VotingKnowledge.PNG}
\caption{Confusion Matrix dla zbioru Knowledge}
\end{figure}

Głosowanie na podstawie dystansu zwykle daje najlepsze wyniki. Metoda \textit{uniform} w przypadku zbioru \textit{Glass} sprawiła, że do jednej z klas nie został zakwalifikowany żaden obiekt. Metody bazujące na dystansie potrafią skorygować sytuacje, gdy liczba obiektów w danej klasie nie jest zbyt duża.

\section{Badanie parametru k}
Dla sposobu pomiaru odległości \textit{uniform} i \textit{distance} zbadano wpływ parametru k dla wszystkich zbiorów przy metodzie liczenia odległości \textit{manhatan} i rozmiarze kroswalidacji 5.

 \begin{tabular}{ |p{2.5cm}||p{2.5cm}|p{2.5cm}|p{2.5cm}|p{2.5cm}| }
\hline
\multicolumn{5}{|c|}{Metoda głosowania uniform}\\
\hline
k & Accuracy & Precision & Recall & FScore \\
\hline
\multicolumn{5}{|c|}{Wine}\\
\hline
2 & 0.955 & 0.958 & 0.955 & 0.955\\
3 & 0.961 & 0.963 & 0.961 & 0.96\\
4 & 0.972 & 0.974 & 0.972 & 0.972\\
5 & 0.966 & 0.968 & 0.966 & 0.966\\
7 & 0.955 & 0.957 & 0.955 & 0.955\\
10 & 0.978 & 0.979 & 0.978 & 0.978\\
15 & 0.961 & 0.963 & 0.961 & 0.96\\
20 & 0.961 & 0.963 & 0.961 & 0.96\\
50 & 0.955 & 0.959 & 0.955 & 0.955\\
\hline
\multicolumn{5}{|c|}{Glass}\\
\hline
2 & 0.626 & 0.637 & 0.626 & 0.628\\
3 & 0.673 & 0.685 & 0.673 & 0.672\\
4 & 0.659 & 0.66 & 0.659 & 0.652\\
5 & 0.701 & 0.7 & 0.701 & 0.69\\
7 & 0.692 & 0.699 & 0.692 & 0.679\\
10 & 0.673 & 0.704 & 0.673 & 0.649\\
15 & 0.664 & 0.702 & 0.664 & 0.639\\
20 & 0.64 & 0.593 & 0.64 & 0.604\\
50 & 0.631 & 0.586 & 0.631 & 0.582\\
\hline
\multicolumn{5}{|c|}{Diabetes}\\
\hline
2 & 0.72 & 0.699 & 0.347 & 0.464\\
3 & 0.727 & 0.626 & 0.537 & 0.578\\
4 & 0.728 & 0.677 & 0.422 & 0.52\\
5 & 0.75 & 0.671 & 0.556 & 0.608\\
7 & 0.732 & 0.657 & 0.485 & 0.558\\
10 & 0.741 & 0.723 & 0.418 & 0.53\\
15 & 0.758 & 0.716 & 0.507 & 0.594\\
20 & 0.758 & 0.741 & 0.47 & 0.575\\
50 & 0.751 & 0.794 & 0.388 & 0.521\\
\hline
\multicolumn{5}{|c|}{Knowledge}\\
\hline
2 & 0.774 & 0.794 & 0.774 & 0.777\\
3 & 0.841 & 0.847 & 0.841 & 0.841\\
4 & 0.821 & 0.833 & 0.821 & 0.822\\
5 & 0.841 & 0.85 & 0.841 & 0.84\\
7 & 0.851 & 0.864 & 0.851 & 0.849\\
10 & 0.836 & 0.852 & 0.836 & 0.834\\
15 & 0.851 & 0.872 & 0.851 & 0.848\\
20 & 0.843 & 0.872 & 0.843 & 0.84\\
50 & 0.751 & 0.805 & 0.751 & 0.738\\
\hline
\end{tabular}
\vspace{5cm}

\begin{tabular}{ |p{2.5cm}||p{2.5cm}|p{2.5cm}|p{2.5cm}|p{2.5cm}| }
\hline
\multicolumn{5}{|c|}{Metoda głosowania distance}\\
\hline
k & Accuracy & Precision & Recall & FScore \\
\hline
\multicolumn{5}{|c|}{Wine}\\
\hline
2 & 0.944 & 0.949 & 0.944 & 0.943\\
3 & 0.961 & 0.963 & 0.961 & 0.96\\
4 & 0.955 & 0.958 & 0.955 & 0.955\\
5 & 0.972 & 0.973 & 0.972 & 0.972\\
7 & 0.955 & 0.957 & 0.955 & 0.955\\
10 & 0.966 & 0.968 & 0.966 & 0.966\\
15 & 0.961 & 0.963 & 0.961 & 0.96\\
20 & 0.966 & 0.968 & 0.966 & 0.966\\
50 & 0.966 & 0.968 & 0.966 & 0.966\\
\hline
\multicolumn{5}{|c|}{Glass}\\
\hline
2 & 0.626 & 0.637 & 0.626 & 0.628\\
3 & 0.673 & 0.685 & 0.673 & 0.672\\
4 & 0.659 & 0.66 & 0.659 & 0.652\\
5 & 0.701 & 0.7 & 0.701 & 0.69\\
7 & 0.692 & 0.699 & 0.692 & 0.679\\
10 & 0.673 & 0.704 & 0.673 & 0.649\\
15 & 0.664 & 0.702 & 0.664 & 0.639\\
20 & 0.64 & 0.593 & 0.64 & 0.604\\
50 & 0.631 & 0.586 & 0.631 & 0.582\\
\hline
\multicolumn{5}{|c|}{Diabetes}\\
\hline
2 & 0.704 & 0.586 & 0.522 & 0.552\\
3 & 0.72 & 0.614 & 0.534 & 0.571\\
4 & 0.725 & 0.624 & 0.534 & 0.575\\
5 & 0.747 & 0.664 & 0.56 & 0.607\\
7 & 0.73 & 0.649 & 0.496 & 0.562\\
10 & 0.75 & 0.692 & 0.511 & 0.588\\
15 & 0.762 & 0.725 & 0.511 & 0.6\\
20 & 0.758 & 0.718 & 0.504 & 0.592\\
50 & 0.755 & 0.767 & 0.429 & 0.55\\
\hline
\multicolumn{5}{|c|}{Knowledge}\\
\hline
2 & 0.801 & 0.804 & 0.801 & 0.802\\
3 & 0.851 & 0.857 & 0.851 & 0.851\\
4 & 0.843 & 0.85 & 0.843 & 0.843\\
5 & 0.856 & 0.864 & 0.856 & 0.855\\
7 & 0.858 & 0.868 & 0.858 & 0.857\\
10 & 0.878 & 0.893 & 0.878 & 0.876\\
15 & 0.871 & 0.887 & 0.871 & 0.869\\
20 & 0.858 & 0.884 & 0.858 & 0.855\\
50 & 0.781 & 0.827 & 0.781 & 0.77\\
\hline
\end{tabular}
\vspace{5cm}

Współczynnik k w przypadku metody głosowania \textit{uniform} powinien być dobierany w taki sposób, aby nie było możliwości dopasowania wektor do takiej samej ilości reprezentantów innych klas. W przypadku metod bazujących na odległości nie ma to aż takiego znaczenia.

\section{Badanie rozmiaru kroswalidacji}
Dla każdego z badanych zbiorów zbadano jak ilość podziałów (a co z tym się wiąże wielkość ciągu uczącego) wpływa na jakość klasyfikacji. Zaimplementowana została kroswalidacja stratyfikowana. Pozostałe parametry zostały ustawione następująco: k=5, metryka="manhatan", sposób głosowania="distance".

\begin{tabular}{ |p{2.5cm}||p{2.5cm}|p{2.5cm}|p{2.5cm}|p{2.5cm}| }
\hline
Folds & Accuracy & Precision & Recall & FScore \\
\hline
\hline
\multicolumn{5}{|c|}{Wine}\\
\hline
2 & 0.944 & 0.949 & 0.944 & 0.943\\
3 & 0.961 & 0.964 & 0.961 & 0.96\\
4 & 0.972 & 0.973 & 0.972 & 0.972\\
5 & 0.972 & 0.973 & 0.972 & 0.972\\
6 & 0.966 & 0.968 & 0.966 & 0.966\\
7 & 0.966 & 0.968 & 0.966 & 0.966\\
8 & 0.972 & 0.973 & 0.972 & 0.972\\
9 & 0.972 & 0.973 & 0.972 & 0.972\\
10 & 0.972 & 0.973 & 0.972 & 0.972\\
\hline
\multicolumn{5}{|c|}{Glass}\\
\hline
2 & 0.593 & 0.57 & 0.593 & 0.575\\
3 & 0.636 & 0.617 & 0.636 & 0.621\\
4 & 0.668 & 0.665 & 0.668 & 0.658\\
5 & 0.701 & 0.7 & 0.701 & 0.69\\
6 & 0.706 & 0.706 & 0.706 & 0.694\\
7 & 0.696 & 0.698 & 0.696 & 0.687\\
8 & 0.682 & 0.683 & 0.682 & 0.676\\
9 & 0.692 & 0.703 & 0.692 & 0.682\\
10 & 0.701 & 0.706 & 0.701 & 0.69\\
\hline
\multicolumn{5}{|c|}{Diabetes}\\
\hline
2 & 0.74 & 0.655 & 0.537 & 0.59\\
3 & 0.75 & 0.674 & 0.549 & 0.605\\
4 & 0.747 & 0.67 & 0.545 & 0.601\\
5 & 0.747 & 0.664 & 0.56 & 0.607\\
6 & 0.732 & 0.641 & 0.526 & 0.578\\
7 & 0.736 & 0.641 & 0.552 & 0.593\\
8 & 0.738 & 0.648 & 0.549 & 0.594\\
9 & 0.738 & 0.648 & 0.549 & 0.594\\
10 & 0.743 & 0.656 & 0.556 & 0.602\\
\hline
\multicolumn{5}{|c|}{Knowledge}\\
\hline
2 & 0.846 & 0.855 & 0.846 & 0.846\\
3 & 0.843 & 0.852 & 0.843 & 0.843\\
4 & 0.856 & 0.865 & 0.856 & 0.855\\
5 & 0.856 & 0.864 & 0.856 & 0.855\\
6 & 0.873 & 0.881 & 0.873 & 0.872\\
7 & 0.861 & 0.87 & 0.861 & 0.86\\
8 & 0.858 & 0.867 & 0.858 & 0.858\\
9 & 0.861 & 0.868 & 0.861 & 0.86\\
10 & 0.856 & 0.864 & 0.856 & 0.856\\
\hline
\end{tabular}

\section{Porównanie wyników między różnymi metodami klasyfikacji}
Dla każdego z algorytmów wybrane zostały optymalne parametry dla każdego ze zbiorów. Do porównania posły nam wyznacznik \textit{F-score}.

\begin{tabular}{ |p{2.5cm}||p{2.5cm}|p{2.5cm}|p{2.5cm}| }
\hline
Data Set & Bayes & C4.5 & knn \\
\hline
Wine & 0.957 & 0.932 & 0.972 \\
Glass & 0.646 & 0.691 & 0.690 \\
Diabetes & 0.748 & 0.816 & 0.608 \\
\hline
\end{tabular}

\begin{figure}[H]
\centering
\includegraphics[width=1\textwidth]{Comparasion.PNG}
\caption{Porównanie działania algorytmów dla trzech zbiorów.}
\end{figure}

\section{Wnioski}
To do..
\end{document}
